% !TEX TS-program = pdflatex
% !TEX encoding = UTF-8 Unicode

%\documentclass[12pt,a4paper]{memoir} % for a long document
\documentclass[12pt,a4paper,article]{memoir} % for a short document

\usepackage[utf8]{inputenc} % set input encoding to utf8

%%% PAGE DIMENSIONS
% Set up the paper to be as close as possible to both A4 & letter:
\settrimmedsize{11in}{210mm}{*} % letter = 11in tall; a4 = 210mm wide
\setlength{\trimtop}{0pt}
\setlength{\trimedge}{\stockwidth}
\addtolength{\trimedge}{-\paperwidth}
\settypeblocksize{*}{\lxvchars}{1.618} % we want to the text block to have golden proportionals
\setulmargins{50pt}{*}{*} % 50pt upper margins
\setlrmargins{*}{*}{1.618} % golden ratio again for left/right margins
\setheaderspaces{*}{*}{1.618}
\checkandfixthelayout 
\usepackage{enumitem}
\setitemize{noitemsep,topsep=0pt,parsep=0pt,partopsep=0pt}
\usepackage{xcolor,listings}
\usepackage{textcomp}
\lstset{upquote=true}
\usepackage{graphicx}

%%% \maketitle CUSTOMISATION
% For more than trivial changes, you may as well do it yourself in a titlepage environment
\pretitle{\begin{center}\sffamily\huge\MakeUppercase}
\posttitle{\par\end{center}\vskip 0.5em}

%%% ToC (table of contents) APPEARANCE
\maxtocdepth{subsection} % include subsections
\renewcommand{\cftchapterpagefont}{}
\renewcommand{\cftchapterfont}{}     % no bold!

%%% HEADERS & FOOTERS
\pagestyle{ruled} % try also: empty , plain , headings , ruled , Ruled , companion

%%% CHAPTERS
\chapterstyle{hangnum} % try also: default , section , hangnum , companion , article, demo

\renewcommand{\chaptitlefont}{\Huge\sffamily\raggedright} % set sans serif chapter title font
\renewcommand{\chapnumfont}{\Huge\sffamily\raggedright} % set sans serif chapter number font

%%% SECTIONS
\hangsecnum % hang the section numbers into the margin to match \chapterstyle{hangnum}
\maxsecnumdepth{subsection} % number subsections

\setsecheadstyle{\Large\sffamily\raggedright} % set sans serif section font
\setsubsecheadstyle{\large\sffamily\raggedright} % set sans serif subsection font

\newlength\drop
\makeatletter
\newcommand*\titleM{\begingroup%
\setlength\drop{0.08\textheight}
\centering
\vspace*{\drop}
{\Huge\bfseries Technical Interview}\\[\baselineskip]
%{\scshape MAN-AHL}\\[\baselineskip]
\vfill
\includegraphics[width=0.5\textwidth]{img/logo.png}\par\vspace{1cm}
\vfill
{\large\scshape Fadil Mokhchane}\par
\vfill
{\scshape \@date}\par
\vspace*{2\drop}
\endgroup}
\makeatother
%% END Memoir customization

\title{Technical Interview}
\author{Fadil Mokhchane}
%\date{ August 2017} % Delete this line to display the current date
%%% BEGIN DOCUMENT
\begin{document}


\begin{titlingpage}
\titleM
\end{titlingpage}

\newpage
\tableofcontents* % the asterisk means that the contents itself isn't put into the ToC
%\clearpage %or \cleardoublepage
\phantomsection
%\listoffigures
%\listoftables
\newpage

\chapter{Algorithm}
\section{Question}
Implement the method nextNum() and a minimal but effective set of unit tests. 
Implement in the language of your choice, Python is preferred, but Java and 
other languages are completely fine. 
Make sure your code is exemplary, as if it was going to be shipped as part of a production system.

As a quick check, given Random Numbers are $[-1, 0, 1, 2, 3]$ and 
Probabilities are $[0.01, 0.3, 0.58, 0.1, 0.01]$ if we call nextNum() 100 times 
we may get the following results. As the results are random, these particular results are unlikely.
\begin{itemize}
	\item -1: 1 times 
	\item 0: 22 times
	\item 1: 57 times 
	\item 2: 20 times
	\item 3: 0 times 
\end{itemize}

\chapter{SQL}
\section{Question}
Given the following tables
\begin{lstlisting}[
           language=SQL,
           showspaces=false,
           basicstyle=\ttfamily,
           numbers=left,
           numberstyle=\tiny,
           commentstyle=\color{gray}
        ]
CREATE TABLE Product
(
product_id number primary key,
name varchar2(128 byte) not null,
rrp number not null,
available_from date not null
);
CREATE TABLE Orders
(
order_id number primary key,
product_id number not null,
quantity number not null,
order_price number not null,
dispatch_date date not null,
foreign key (product_id) references Product(product_id)
);
\end{lstlisting}
Write an sql query to find books that have sold fewer than 10 copies in the last year, 
excluding books that have been available for less than 1 month.

Some example data, intended to give an idea of the sort of data that might exist, 
can be found below. Note that the data is not complete, 
nor does it necessarily cover all the cases that might be encountered.

\begin{center}
    \begin{tabular}{| l | l | l | l |}
    \hline
	product\_id & name & rrp & available\_from \\
	\hline 
	101 & Bayesian Methods for ...& 94.95 & (last thursday) \\
	102 & (next year) in Review (preorder) & 21.95 & (next year) \\
	103 & Learn Python in Ten Minutes & 2.15 & (three months ago) \\
	104 & sports almanac (1999-2049) & 3.38 & (2 years ago) \\
	105 & finance for dummies & 84.99 & (1 year ago)  \\
	\hline
    \end{tabular}
\end{center}
\begin{center}
    \begin{tabular}{| l | l | l | l | l |}
    \hline
	order\_id & product\_id & quantity & order\_price & dispatch\_date \\
	\hline 
	1000 & 101 & 1 & 90.00 & (two months ago) \\
	1001 & 103 & 1 & 1.15 & (40 days ago) \\
	1002 & 101 & 10 & 90.00 & (11 months ago) \\
	1003 & 104 & 11 & 3.38 & (6 months ago) \\
	1004 & 105 & 11 & 501.33 & (two years ago) \\
	\hline
    \end{tabular}
\end{center}

%\newpage
%\appendix
%\addcontentsline{toc}{chapter}{Appendices}
%\chapter{Installing the App}

\end{document}
